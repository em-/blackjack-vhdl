\documentclass [11pt,a4paper,oneside,draft]{article}

\usepackage[italian]{babel}
\usepackage[utf8]{inputenc}
\usepackage{indentfirst}
\usepackage{graphicx}
\usepackage{geometry}
\usepackage{verbatim}

% Informations
\title{Implementazione del gioco del blackjack in VHDL}
\author{Emanuele Aina (matr. 129548)}
\date{}

% Margins
\geometry{bindingoffset=0.5cm}

% Caption
\newcommand{\mc}[1]{{\small\ #1}}
\newcommand{\makecaption}[1]{\caption{\mc\ #1}}

% URL
\newcommand{\url}[2]{#1\footnote{#2}}

% References
\newcommand{\refemph}[1]{\emph{#1}}
\newcommand{\imgref}[1]{\refemph{\ref{img:#1}}}
\newcommand{\imgrefx}[1]{\refemph{figura}~\imgref{#1}}

% Images
\newcommand{\im}[3]{
    \begin{figure}[!htb]
        \begin{center}
            \includegraphics{img/#1}
            \makecaption{#2}
            \label{img:#3}
        \end{center}
    \end{figure}
}

% Ignore the argument
\newcommand{\noop}[1]{}

% Foreign words
\newcommand{\foreign}[1]{\emph{#1}}

% Justify even if it has to leave a lot of blank space
\setlength{\emergencystretch}{3em}


\begin{document}

\maketitle

\begin{center}
\small{01GSS - Specifica e simulazione dei sistemi digitali}
\end{center}

\section{Specifiche di progetto}
Si progetti, a livello RT, un circuito che permetta di giocare al gioco del Blackjack.

\subsection{Spiegazione del gioco}
Il gioco si svolge tra due giocatori, di seguito denominati
Giocatore e Banco, con un mazzo di carte da poker. Inizia a giocare il Giocatore,
decidendo se estrarre una carta dal mazzo, oppure passare la mano al Banco. Ad ogni
estrazione, il valore della carta estratta viene sommata al punteggio corrente: lo scopo
è arrivare a totalizzare un punteggio il più vicino possibile a 21, senza mai superarlo.
Se tale valore viene superato durante la mano del Giocatore, il Banco vince
immediatamente. Viceversa, il Banco comincia ad estrarre carte dal mazzo con lo
scopo di realizzare un punteggio almeno pari a quello del Giocatore, senza comunque
superare il limite di 21. Se ciò non accade, vince il Giocatore.

\subsection{Implementazione}
Il circuito possiede quattro ingressi di controllo Reset, NewGame,
Stop ed En di parallelismo pari ad 1 bit e un ingresso di dato I di parallelismo pari a
3 bit, e si deve comportare nel seguente modo:

\begin{itemize}
\item Se Reset = 1, il circuito si porta sempre nello stato iniziale, in cui tutti i led e le
      4 cifre del display a 7 segmenti sono spenti, indipendentemente dal valore
      degli altri segnali.

\item La prima volta che il segnale NewGame assume il valore 1, il sistema si
      prepara a giocare una nuova partita: le due cifre più a destra del display a 7
      segmenti si accendono, riportando il valore del punteggio iniziale (ovvero 00)
      del Giocatore, mentre le due cifre più a sinistra del display e i led sono spenti.

\item Dopo aver iniziato una partita, ogni volta che En assume il valore 1, i segnali
      dell'ingresso di dato I vengono campionati e memorizzati. Essi rappresentano
      il valore, in binario, della carta estratta dal mazzo (si utilizzano carte con valori
      limitati tra 1 e 8, dove il valore 8 è codificato su tre bit come 000). Tale valore
      deve essere visualizzato sui led, ed essere sommato al punteggio corrente.

\item Nel momento in cui il segnale Stop si porta a 1, la mano passa al Banco: le
      due cifre più a sinistra del display a 7 segmenti si accendono e riportano il
      valore iniziale del punteggio del Banco (di nuovo, 00). Analogamente al passo
      precedente, ogni volta che il segnale assume il valore 1, una carta viene
      estratta dal mazzo, il suo valore viene visualizzato sui led e sommato al
      punteggio del Banco.

\item La partita termina quando:
      \begin{itemize}
      \item Il Banco ha totalizzato un punteggio superiore a quello del Giocatore,
            senza eccedere il limite di 21. In questo caso, i punti decimali relativi
            alle due cifre del punteggio del Banco si accendono.

      \item Il Banco ha totalizzato un punteggio superiore al limite di 21: in questo
            caso, si illuminano i punti decimali relativi alle due cifre che riportano il
            punteggio del Giocatore.

      \item Durante la sua mano, il Giocatore ha ottenuto un punteggio superiore a
            21: come nel primo caso, si illuminano i punti decimali delle due cifre
            del punteggio del Banco. Si noti che in questo caso il Banco non gioca
            affatto, quindi i punti decimali si devono accendere, ma le due cifre del
            display a 7 segmenti devono rimanere spente.
      \end{itemize}

\item A questo punto, il sistema si ferma e aspetta che il segnale NewGame si riporti
      a 1, in modo da iniziare una nuova partita.
\end{itemize}

Si implementi il circuito richiesto sulla FPGA in dotazione, utilizzando i bottoni BTN3,
BTN2, BTN1 e BTN0 per i segnali di Reset, NewGame, Stop ed En, e gli switch
SW7, SW6 e SW5 per il segnale di dato I. Il circuito lavori ad una frequenza di clock
pari a 50 kHz.

Si osservi che:
\begin{itemize}
\item Il numero visualizzato sul display deve essere codificato in base 10 (quindi, se
      il suo valore è 19 e viene incrementato, il nuovo valore da visualizzare è
      effettivamente 20, non 1A).

\item I led da illuminare ogni volta che una carta viene estratta sono in numero pari
      al valore della carta, e partono sempre da LD0. Per esempio, se la carta
      estratta ha un valori pari a 3, si devono illuminare i led LD0, LD1 e LD2.
\end{itemize}


\section{Implementazione}
\subsection{Metodologia}
Il progetto è stato implementato esclusivamente mediante descrizioni VHDL, sia a
livello comportamentale che a livello strutturale.

Lo sviluppo è avvenuto in ambiente Linux (Debian etch) utilizzando GHDL 0.25 per le
simulazioni e, successivamente, in ambiente Windows con Xilinx 6.1i per la
programmazione su FPGA.

Per ogni componente sviluppato è stato inoltre realizzato un testbench in VHDL 
puro, in modo da renderne possibile la verifica in modo automatico (make run). %FIXME

\subsection{Board}
Il componente ``board'' rappresenta l'entità di massimo livello e costituisce
la parte del progetto che verrà programmata sulla FPGA.

\verbatiminput{board/board.vhdl}

La descrizione è strutturale e provvede a istanziare i componenti ``blackjack''
e ``display'' connettendoli tra loro e a impiegare ``clock\_divider'' per 
convertire il clock della scheda a frequenze inferiori per la logica di 
gioco e per la logica di gestione del display. I segnali di input vengono 
inoltre filtrati usando dei generatori di impulsi al fine di evitare 
campionamenti multipli di una singola pressione.

Il componente è dotato del seguente testbench per la verifica automatica.

\verbatiminput{board/tb_board.vhdl}

Oltre all'unità da verificare, viene anche istanziato un simulatore per il
sistema di gestione dei display a sette segmenti disponibili sulla FPGA,
controllando il funzionamento della logica di codifica e multiplexing.

Il clock viene generato con un periodo di 1ns fino a che non esistano più
eventi per la simulazione. A tal punto anche il clock viene fermato in 
modo da segnalare al simulatore la fine del test.

Il processo ``count'' si limita a conteggiare il numero di colpi di clock
in modo che il processo ``test'' possa impostare i vettori di test con 
la cadenza specificata. Questo processo, infatti, si occupa di leggere
i vettori di input e gli output attesi da un file e applicarli all'unità
testata. Nel caso specifico, dovendo simulare l'input dell'utente, la
frequenza con cui gli input vengono applicati è nell'ordine dei 
millisecondi anziché dei nanosecondi usati dal clock di simulazione.

\verbatiminput{board/tb_board.test}

Su ogni riga è specificato l'istante corrispondente, gli input e gli
output.

\end{document}
